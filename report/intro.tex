Analysis of political discourse is a hot topic. In 2008, Nate Silver made national headlines by releasing a collection of statistically-backed predictions for the US presidential election by corerctly predicting the winner of 49 of 50 states \cite{Silver} 

Since then, the popularity of statistical and machine learning analysis of polls has emerged as a more seriously regarded task and reinvigorated the investigation into political discourse using other statistical approaches.\\

Amber Boydstun and Jeff Korn developed a smart phone application that led to a corpus over the 2012 presidential debates\cite{Boydstun}
Consider Latent Dirichlet Allocation (LDA): developed in 2003, LDA is a generative model that allows a set of documents to be explained by some unobserved set of 'topics' that explain the observed words in a document. By viewing the concatenation of such responses, they can be used to generate a probabilistic distribution of topics that give a generative model of the responses in each document.

This report is primarily investigative: LDA is used to generate topics based on predetermined clusters(Users, Hot Topics: Economy, etc.) Once generated, the second half of this report will concern to interpretation of these topics.